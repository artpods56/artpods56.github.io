%%%%%%%%%%%%%%%%%%%%%%%%%%%%%%%%%%%%%%%
% Deedy - One Page Two Column Resume
% LaTeX Template
% Version 1.2 (16/9/2014)
%  
% Original author:
% Debarghya Das (http://debarghyadas.com)
%
% Original repository:
% https://github.com/deedydas/Deedy-Resume
%
% IMPORTANT: THIS TEMPLATE NEEDS TO BE COMPILED WITH XeLaTeX
%
% This template uses several fonts not included with Windows/Linux by
% default. If you get compilation errors saying a font is missing, find the line
% on which the font is used and either change it to a font included with your
% operating system or comment the line out to use the default font.
% 
%%%%%%%%%%%%%%%%%%%%%%%%%%%%%%%%%%%%%%

\documentclass[]{deedy-resume-openfont}
\usepackage{fancyhdr}
\usepackage{tikzpagenodes}
\usepackage[document]{ragged2e}





\pagestyle{fancy}
\fancyhf{}

\begin{document}

\begin{tikzpicture}[remember picture,overlay,shift={(current page.north west)}]
\node[anchor=north west,xshift=1.75cm,yshift=-0.25cm]{\includegraphics[width=2.5cm]{apods_img.jpg}};
\end{tikzpicture}
%%%%%%%%%%%%%%%%%%%%%%%%%%%%%%%%%%%%%%
%
%     LAST UPDATED DATE
%
%%%%%%%%%%%%%%%%%%%%%%%%%%%%%%%%%%%%%%
\lastupdated

%%%%%%%%%%%%%%%%%%%%%%%%%%%%%%%%%%%%%%
%
%     TITLE NAME
%
%%%%%%%%%%%%%%%%%%%%%%%%%%%%%%%%%%%%%%
\namesection{Artur}{Podsiadły}{
    \href{mailto:artpods56@gmail.com}{\faEnvelopeO\ artpods56@gmail.com}\hspace{0,25cm} 
    \href{https://www.linkedin.com/in/artpods56/}{\faLinkedin\ linkedin.com/in/artpods56} \\
    \href{https://github.com/artpods56}{\faGithub\ github.com/artpods56} \hspace{0,25cm} 
    \faMobile{ +48 530-670-682} 
    \vspace{0.2cm}
}



%%%%%%%%%%%%%%%%%%%%%%%%%%%%%%%%%%%%%%
%
%     COLUMN ONE
%
%%%%%%%%%%%%%%%%%%%%%%%%%%%%%%%%%%%%%%

\begin{minipage}[t]{0.4\textwidth} 

%%%%%%%%%%%%%%%%%%%%%%%%%%%%%%%%%%%%%%
%     OBJECTIVE
%%%%%%%%%%%%%%%%%%%%%%%%%%%%%%%%%%%%%%

\section{O MNIE}
{Pasjonuje się szeroko pojętą sztuczną inteligencją, z silnym naciskiem na MLOps/LLMOps i systemy RAG. Specjalizuję się w projektowaniu i wdrażaniu kompleksowych potoków ML, automatyzacji procesów wdrażania oraz integracji rozwiązań LLM w środowiskach produkcyjnych.}

%%%%%%%%%%%%%%%%%%%%%%%%%%%%%%%%%%%%%%
%     CERTIFICATIONS
%%%%%%%%%%%%%%%%%%%%%%%%%%%%%%%%%%%%%%

\section{Certyfikaty}
\textbf{Świadectwo kwalifikacji EE.09} \\
Programowanie, tworzenie i administrowanie stronami internetowymi i bazami danych. \\

\textbf{Świadectwo kwalifikacji EE.08} \\
Montaż i eksploatacja systemów komputerowych, urządzeń peryferyjnych i sieci. \\

\textbf{Cisco Networking Academy - IT Essentials} \\
Certyfikat potwierdzający teoretyczne i praktyczne umiejętności zawodowe. \\

%%%%%%%%%%%%%%%%%%%%%%%%%%%%%%%%%%%%%%
%     COURSEWORK
%%%%%%%%%%%%%%%%%%%%%%%%%%%%%%%%%%%%%%

\section{Wybrane kursy}
\subsection{Studia licencjackie}
Podstawy uczenia maszynowego \\
Głębokie sieci neuronowe w przetwarzaniu danych
Przetwarzanie języka naturalnego \\
Głębokie sieci neuronowe w wizji komputerowej

%%%%%%%%%%%%%%%%%%%%%%%%%%%%%%%%%%%%%%
%     SKILLS
%%%%%%%%%%%%%%%%%%%%%%%%%%%%%%%%%%%%%%

\section{Umiejętności}
\subsection{Development}
Python (SOLID, Clean Code, TDD) \textbullet{} FastAPI \textbullet{} Django \textbullet{} Git \textbullet{} PostgreSQL
\sectionsep

\subsection{Data i AI}
HuggingFace Transformers | Models | Datasets \textbullet{} PyTorch \textbullet{} MCP \textbullet{} Semantic Kernel \textbullet{} LangGraph \textbullet{} RAG \textbullet{}  Vision LLMs \textbullet{} DSPy
\sectionsep

\subsection{Ops i Usługi}
Docker \textbullet{} llama.cpp \textbullet{} Dagster \textbullet{} Label Studio \textbullet{} MinIO \textbullet{} Weights\&Biases \textbullet{} OpenRouter \textbullet{} Hydra

%%%%%%%%%%%%%%%%%%%%%%%%%%%%%%%%%%%%%%
%     PROJECTS
%%%%%%%%%%%%%%%%%%%%%%%%%%%%%%%%%%%%%%

\section{Projekty}
\runsubsection{\href{https://github.com/mereolog/ml-playground}{ML Playground}}
\descript{Interaktywna platforma do uczenia maszynowego}
\location{Python | Pydantic | Django | FastAPI | Docker}

\sectionsep
\runsubsection{\href{https://github.com/artpods56/AlphaBetaLogic}{AlphaBetaLogic}}
\descript{Biblioteka Pythona do parsowania i analizowania wyrażeń logiki pierwszego rzędu}
\location{Python | PLY (Python Lex-Yacc) | NetworkX}


%%%%%%%%%%%%%%%%%%%%%%%%%%%%%%%%%%%%%%
%     COLUMN TWO
%
%%%%%%%%%%%%%%%%%%%%%%%%%%%%%%%%%%%%%%

\end{minipage} 
\hfill
\begin{minipage}[t]{0.55\textwidth} 

\section{Edukacja} 
\subsection{Katolicki Uniwersytet Lubelski}
\subsection{Jana Pawła II}
\descript{Licencjat ze Sztucznej Inteligencji \\ Praca dyplomowa: "Ewolucja systemów RAG (Retrieval-Augmented Generation)"}
\location{Paź 2022 | Cze 2025 | Ukończono}
\sectionsep 

\subsection{Zespół Szkół Energetycznych im. prof. Kazimierza Drewnowskiego w Lublinie}
\descript{Specjalizacja Technik Informatyk}
\location{2017 | 2021}
\sectionsep



%%%%%%%%%%%%%%%%%%%%%%%%%%%%%%%%%%%%%%
%     EXPERIENCE
%%%%%%%%%%%%%%%%%%%%%%%%%%%%%%%%%%%%%%

\section{Doświadczenie}

\runsubsection{Ośrodek Badań nad Geografią Historyczną Kościoła w Polsce} \\
\descript{Architekt Rozwiązań | Inżynier MLOps/LLMOps}
\location{Marzec 2025 - Sierpień 2025 | Lublin}
\vspace{\topsep} % Hacky fix for awkward extra vertical space
\begin{tightemize}
\item Zaprojektowałem i zbudowałem kompleksowe potoki ML do automatycznej ekstrakcji informacji z historycznych schematyzmów przy użyciu LayoutLMv3 i wizyjnych LLM.
\item System osiągnął ponad 90\% dokładności w ekstrakcji danych strukturalnych ze złożonych układów dokumentów i jest przygotowywany do wdrożenia produkcyjnego.
\item Skonfigurowałem gotową do produkcji platformę do adnotacji z użyciem Dockera, Label Studio i MinIO, stosując najlepsze praktyki MLOps, w tym wersjonowanie modeli i śledzenie eksperymentów.
\end{tightemize}
\sectionsep

\runsubsection{TEDxLublin} \\
\descript{IT / Projektowanie i Rozwój Stron Internetowych}
\location{Marzec 2024 - Obecnie | Lublin}

\begin{tightemize}
\item Zaprojektowałem i wdrożyłem oficjalną stronę internetową TEDxLublin która pozwala zarządzać wydarzeniami.
\item Zautomatyzowałem formularze wolontariuszy, prelegentów i partnerów, a także subskrypcję newslettera.
\item Zintegrowałem Umami do prywatnej analityki internetowej, zapewniając zgodność z RODO.
\end{tightemize}
\sectionsep

\runsubsection{Katolicki Uniwersytet Lubelski Jana Pawła II} \\
\descript{Badania i Rozwój AI / Rozwój Oprogramowania}
\location{Wrzesień 2022 - Luty 2023 | Lublin}
\begin{tightemize}
\item Prowadziłem badania i rozwój w zakresie technik przetwarzania języka naturalnego oraz architektury RAG na potrzeby rozwoju asystenta/chatbota AI w środowisku akademickim.
\item Analizowałem i oceniałem skuteczność różnych modeli i podejść AI w kontekście zadań QA.
\end{tightemize}
\sectionsep

\end{minipage} 

\vfill % This ensures the clause is pushed to the bottom of the document if there's space
\begin{center}
    \scriptsize % Makes the font smaller
    Wyrażam zgodę na przetwarzanie moich danych osobowych zawartych w moim CV, zgodnie z RODO (UE) 2016/679, art. 6 ust. 1 lit. a.
\end{center}

\end{document}
